\documentclass[a4paper,10pt]{article}
\usepackage{amsmath}
\usepackage{amsfonts}
\usepackage{graphicx}
\usepackage{color}

% If you want to use Chinese, include the following package
\usepackage{CJKutf8}

\title{CS532100 Numerical Optimization Homework 2}
\author{Student ID name}
\date{Due Dec 10}
\begin{document}
\maketitle
\begin{enumerate}
 \item Consider the linear least square problem:$$min_{\vec{x}\in \mathbb{R}^2}||A\vec{x} - \vec{b}||^2 ,$$ where $$A = \left[ \begin{array}{cc}
    4 & 8 \\ 2 & 4 \\ 1 & 2
    \end{array} \right], \vec{b} = \left(\begin{array}{c} 21/4 \\ 0 \\ 0 \end{array} \right)$$

\begin{enumerate}
		\item (10\%) Write its normal equation.


{\color{blue} Answers are put here. 

    \begin{CJK*}{UTF8}{bsmi}

\end{CJK*}

}



        \item (10\%) Express $\vec{b} = \vec{b}_1 + \vec{b}_2$ such that $\vec{b}_1$ is in the subspace spanned by $A$'s
column vectors, and $\vec{b}_2$ is orthogonal to $A$'s column vectors.

{\color{blue} Answers are put here. 

    \begin{CJK*}{UTF8}{bsmi}
    
\end{CJK*}

}

        \item (10\%) Show that $\vec{z}\in \mathbb{R}^2$ is a least square solution for $A\vec{x}=\vec{b}$ if and only if $\vec{z}$ is part of a solution to the larger linear system:
$$\left[ \begin{array}{cc}
    0 & A^T \\ A & I
    \end{array} \right] \left[\begin{array}{cc} \vec{z} \\ \vec{y} \end{array}\right] = \left[\begin{array}{cc}  0\\ \vec{b}\end{array}\right]$$


{\color{blue} Answers are put here. 

    \begin{CJK*}{UTF8}{bsmi}

\end{CJK*}
}


\end{enumerate}
    \item In Note05 (Page 16), memoryless BFGS iteration matrix $H_{k+1}$ can be derived from considering the Hestenes–Stiefel form of the nonlinear conjugate gradient method. Recalling that $\vec{s}_k = \alpha_k \vec{p}_k$, we have that the search direction for this method
is given by
$$\vec{p}_{k+1} = -\nabla f_{k+1} + \frac{\nabla f_{k+1}^T\vec{y}_k}{\vec{y}^T\vec{p}_k}\vec{p}_k $$ $$ = -\nabla f_{k+1} + \frac{\nabla f_{k+1}^T\vec{y}_k}{\vec{y}^T\vec{s}_k}\vec{s}_k $$ $$ = -( I - \frac{\vec{s}_k\vec{y}_k^T}{\vec{y}^T\vec{s}_k})\nabla f_{k+1} $$ $$ = - \hat{H}_{k+1} \nabla f_{k+1}$$
However, the matrix $\hat{H}_{k+1}$ is neither symmetric nor positive definite.
\begin{enumerate}
    \item (10\%) Please show that the matrix $\hat{H}_{k+1}$ is singular. $\\$ (You can only prove it for the case $\nabla f_k, \vec{p}_k, \vec{y}_k, \vec{s}_k \in \mathbb{R}^2$ for all $k \in \mathbb{N}$.)
{\color{blue} $\\$ Answers are put here. 

\begin{CJK*}{UTF8}{bsmi}

\end{CJK*}

}
    \item (0\%) Please read the reference book (Page 180) to understand the derivation of the inverse hessian formula in Note05 (Page 16). $\\$(you don't need to write anything in this subproblem.) $$H_{k+1} = (I - \frac{\vec{s}_k\vec{y}_k^T}{\vec{y}_k^T\vec{s}_k})(I - \frac{\vec{y}_k\vec{s}_k^T}{\vec{y}_k^T\vec{s}_k}) + \frac{\vec{s}_k\vec{s}_k^T}{\vec{y}_k^T\vec{s}_k}$$
\end{enumerate}

\item (10\%) The total least square problem is to solve the following problem
$$\min_{\vec{x}, \|\vec{x}\|=1} \vec{x}^TA^TA\vec{x}$$
where $A$ is an $m\times n$ matrix.  Here we assume $m>n$.  
Let $A=U\Sigma V^T$ be the SVD of $A$, where $U$ is the matrix of left singular vectors, $V$ is the matrix of right singular vectors, and $\Sigma$ is a diagonal matrix with diagonal elements
$\sigma_1, \sigma_2, \ldots, \sigma_n$.  Moreover, $U$ and $V$ are orthogonal matrices, and $\sigma_1\ge \sigma_2 \ge \cdots \ge \sigma_n$.
Show the solution of the above problem is the $\sigma_n^2$.

{\color{blue} Answers are put here. 

\begin{CJK*}{UTF8}{bsmi}


\end{CJK*}

}

\item Consider the following linear programming problem:
$$\begin{array}{lll}
        \max_{x_1,x_2} & z=x_1+x_2 \\
        \mbox{s.t.} & x_1 + 2x_2 \le 4  \\
         & 4x_1 + 2x_2 \le 12   \\
         & -x_1 + x_2 \le 1   \\
         & x_1, x_2 \ge 0
      \end{array}$$

\begin{enumerate}
    \item (10\%) Please refer Note08 (Page 2) to draw the figure of the constraints by any means, and use that to solve the problem. 
{\color{blue} $\\$Answers are put here. 

    \begin{CJK*}{UTF8}{bsmi}

\end{CJK*}
}
    \item (10\%) Derive its dual problem and solve the dual problem by any means.
Compare the solutions of the primal and the dual problems.
{\color{blue} Answers are put here. 

    \begin{CJK*}{UTF8}{bsmi}

\end{CJK*}
}
    \item (10\%) Verify the complementarity slackness condition.
{\color{blue} $\\$ Answers are put here. 

    \begin{CJK*}{UTF8}{bsmi}

\end{CJK*}
}
    \item (10\%) Transform the problem to the standard form.
{\color{blue} $\\$ Answers are put here. 

    \begin{CJK*}{UTF8}{bsmi}

\end{CJK*}
}
    \item (10\%) Solve it by the simplex method, as provided in Figure 1, using $\vec{x}_0 = (0, 0)$.
    Indicate $B_k, N_k, \vec{s}_k, \vec{d}_k, p_k, q_k, \gamma_k$ in each step.\
{\color{blue} $\\$ Answers are put here. 

    \begin{CJK*}{UTF8}{bsmi}

\end{CJK*}
}
\end{enumerate}

\begin{figure}[hb]
  \begin{center}
  \begin{tabular}{cp{.2in}l}
    \hline \hline \\[0pt]
  (1) & \multicolumn{2}{l}{Given a basic feasible point $\vec{x}_0$ and the corresponding index set}\\
   & \multicolumn{2}{l}{ $\mathcal{B}_0$ and $\mathcal{N}_0$.} \\
  (2) & \multicolumn{2}{l}{For $k=0,1,\ldots$} \\
  (3) & & Let $B_k=A(:,\mathcal{B}_k), N_k=A(:,\mathcal{N}_k)$, $\vec{x}_B=\vec{x}_k(\mathcal{B}_k),
  \vec{x}_N=\vec{x}_k(\mathcal{N}_k)$, \\
     &  & and $\vec{c}_B=\vec{c}_k(\mathcal{B}_k), \vec{c}_N=\vec{c}_k(\mathcal{N}_k)$.\\
  (4) & & Compute $\vec{s}_k=\vec{c}_N-N_k^T(B_k^{-1})^T\vec{c}_B$  \mbox{\color{blue}(pricing)} \\
  (5) & & If $\vec{s}_k\ge 0$, return the solution $\vec{x}_k$. \mbox{\color{blue}(found optimal solution)} \\
  (6) & & Select $q_k\in\mathcal{N}_k$ such that $\vec{s}_k(i_q)<0$, \\
    & & where $i_q$ is the index of $q_k$ in $\mathcal{N}_k$\\
  (7) & & Compute $\vec{d}_k=B^{-1}_kA_k(:,q_k)$. \mbox{\color{blue}(search direction)} \\
  (8) & & If $\vec{d}_k\le 0$, return \verb"unbounded". \mbox{\color{blue}(unbounded case)} \\
  (9) & & Compute $\displaystyle [\gamma_k, i_p] = \min_{i,\vec{d}_k(i)>0}\frac{\vec{x}_B(i)}{\vec{d}_k(i)}$ \mbox{\color{blue}(ratio test)}\\
  &&\mbox{\color{blue}(The first return value is the minimum ratio;}\\
  &&\mbox{\color{blue} the second return value
  is the index of the minimum ratio.)} \\
  (10) & & $\displaystyle x_{k+1}\left(
                     \begin{array}{c}
                       \mathcal{B} \\
                       \mathcal{N} \\
                     \end{array}
                   \right) =  \left(
                     \begin{array}{c}
                       \vec{x}_B \\
                       \vec{x}_N \\
                     \end{array}
                   \right) + \gamma_k \left(
                     \begin{array}{c}
                       -\vec{d}_k \\
                       \vec{e}_{i_q} \\
                     \end{array}
                   \right)$ \\
  &&\mbox{\color{blue}($\vec{e}_{i_q}=(0,\ldots,1,\ldots,0)^T$ is a unit vector with $i_q$th element 1.)}\\
  (11) &&Let the $i_p$th element in $\mathcal{B}$ be $p_k$. \mbox{\color{blue}(pivoting)} \\
       & & $\mathcal{B}_{k+1}=(\mathcal{B}_k-\{p_k\})\cup\{q_k\}$,  $\mathcal{N}_{k+1}=(\mathcal{N}_k-\{q_k\})\cup\{p_k\}$  \\[10pt]
  \hline \hline
  \end{tabular}
  \end{center}
  \caption{The simplex method for solving (minimization) linear programming}\label{}
\end{figure}

\end{enumerate}

\end{document}